\section{Introduction}
\par  Battery powered electric motors offer many benefits over the internal combustion engine \cite{Mahmoud2016} such as reduced maintenance \cite{poornesh_comparative_2020}, zero emissions \cite{kato_comparative_2013}, and access to renewable energy \cite{cheng_smart_2020}, which have caused many transit authorities to adopt battery powered electric buses (BEBs). 
\par Despite their benefits, the transition to BEBS must address the challenge of extended refuel times. When a bus fueled by diesel or compressed natural gas (CNG) runs low on fuel, the bus may refuel in five to ten minutes, whereas an electric bus may require several hours, presenting logistical challenges for bus fleets. Therefore, maintaining a route schedule while staying charged is a primary concern that BEBs face, and requires careful planning that models how batteries discharge along routes, how long BEBs must charge, and limitations on the number of chargers. 
\par One way in which charge times may be reduced is by charging a bus while it is in motion through dynamic charging. There are a number of ways to do this, including overhead \cite{csonka_optimization_2021} and inductive charging \cite{jeong_automatic_2018} \cite{balde_electric_2019}. An overhead charging scenario allows the bus to charge on overhead power lines while in motion. Inductive charging relies on specialized hardware in the roads that transfers energy to buses that pass overhead. Both methods remove the need to stop for service and allow an electric vehicle to stay in service indefinitely. Unfortunately, both methods require extensive infrastructure \cite{Alwesabi_2022_Robust} that may not be available, or is cost prohibitive to install.  
\par In the absence of infrastructure, \cite{jain_battery_2020} and \cite{xian_zhang_optimal_2016} have proposed methods that exchange depleted batteries for fresh ones. Such a method would eliminate both the logistical challenges of planning and the infrastructure dependence of dynamic charging. The only drawback, is that BEBs are not built with battery exchanges in mind, therefore the task can require specialized hardware, technical expertise, or automation, all of which add complexity and cost.  
\par One charge option that avoids both the infrastructure demands of dynamic charging and the technical difficulties of battery swapping is stationary charging, which plans rest periods into a bus's schedule during which that bus can charge. Stationary charging is the least invasive form of bus charging because it only requires charging hardware at specific locations and makes no exchanges to bus batteries. Prior work in this area addresses a number of problems, including distributed charging networks \cite{Nimalsiri2020}, bus availability, environmental impact \cite{zhou_bi-objective_2021}, route scheduling \cite{Rinalde_Mixed_2020}, battery health \cite{houbbadi_optimal_2019}, the cost of electricity \cite{Leou_optimal_2017}, and the cost of charging infrastructure \cite{Wei2018}.
\par One drawback to using a stationary charging solution is that it does require significant rest periods for charging. One way to decrease the charge intervals is to use high power chargers, which deliver more energy in a smaller period of time. However doing so places large power demands on electrical infrastructure \cite{stahleder_impact_2019} which may result in problems with network reliability \cite{deb_impact_2017} and require additional maintenance and upgrades, which increase the cost of energy \cite{boonraksa_impact_2019}. An effective charge plan must therefore balance the need to charge quickly with the desire to maintain a low power profile \cite{ojer_development_2020}.
\par The authors of \cite{qin_numerical_2016} and \cite{Wang2019} propose simple, heuristic approaches to reduce power demands from BEB fleets. Work done by \cite{bagherinezhad_spatio-temporal_2020} uses a mixed integer linear program (MILP) to solve for a solution, which addresses both when buses should charge, and where they should deploy. Finally, a paper by \cite{He_2019_Fast} provides a MILP framework for minimizing the cost of demand power and both \cite{He_2022_Battery} and \cite{Liu_2022_Optimal} minimize the cost from time of use tariffs. Each of the aforementioned methods focus on demand power in relation to electric bus fleets, but do not account for external activity on the grid, such as effects from electric trains, renewable energy devices, or other utilities which we refer to as ``uncontrolled loads''.  
\par In this paper, the uncontrolled load profile comes from historical data provided by the Utah Transit Authority in Salt Lake City which describes the power demands for an electric train as it passes through the station. In practice, buses would share a single meter with the train. If buses were to charge at high rates while the train drew power from the grid to accelerate, the resulting 15-minute average power would become significant, increasing the monthly cost. 
\par This paper considers a traditional scenario where each bus begins the day in the station and spends the day either on-route or in the station. Buses on route are considered unavailable and cannot charge until that bus returns to the station. For such a scenario, we develop a planning method to manage bus charging by viewing the charge problem in a bin packing context \cite{Ma_Mixed-integer_2017} in a way that minimizes the joint power use from the bus fleet and uncontrolled loads while yielding a precise time schedule for charging.
\par The rest of this paper is organized as follows: Section \ref{sec:4_formulation} discusses the basic problem formulation, Section \ref{sec:5_battery} discusses linear constraints that govern the behavior and limitations of the rate of charge. Section \ref{sec:uncontrolled} discusses how to incorporate uncontrolled loads into the optimization framework. Section \ref{sec:objective} explains how the objective function is formed, and Section \ref{sec:results} discusses performance.
