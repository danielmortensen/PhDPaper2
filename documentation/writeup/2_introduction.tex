\section{Introduction}
\par  Battery electric buses (BEBs) are becoming more popular because of their reduced maintenance costs \cite{poornesh_comparative_2020}, zero emissions \cite{kato_comparative_2013}, and access to renewable energy \cite{cheng_smart_2020}.
\par  By themselves however, BEBs are not a sufficient replacement for diesal or CNG buses because of the difference in refuel times. Refueling Diesal and CNG buses is inherently easy because they can refuel in several minutes. BEBs however are more challenging as they require up to several hours for a full charge. This difference presents an unconventional logistical challenge that is present in traditional bus operations. 
\par The charge time can be reduced by increasing the energy transfer rate, or power.  However, large power demands require costly upgrades to the existing power infrastructure, increasing the overall cost of energy \cite{stahleder_impact_2019}, \cite{deb_impact_2017}, \cite{boonraksa_impact_2019}. Therefore, minimising the cost of energy requires a low power profile.
\par Additionally, the power profile also depends on non-BEB demands, or uncontrolled loads. Because other appliances and utilities also affect the load profile, their combined use with BEBs can significantly increase the cost of power. 
\par Prior work has focused on optimising day to day operations for a variety of scenarios including dynamic overhead charging \cite{csonka_optimization_2021}, dynamic inductive charging \cite{jeong_automatic_2018} \cite{balde_electric_2019}, battery swapping \cite{jain_battery_2020} \cite{xian_zhang_optimal_2016}, and stationary charging \cite{whitaker_network_nodate}.
\par each hardware type has its merits. Both the dynamic overhead and inductive charging eliminate the need for planning and down time as buses are charged while in service, but require expensive infrastructure that may not be feasible to install or purchase.
\par battery swapping doesn't require any type of hardware installation and removes limitations imposed by route schedules, but requires specialized tools and/or automation.
\par stationary charging is the least invasive form of bus charging because it only requires charging hardware at specific locations and makes no changes to bus batteries. 
\par stationary charging is most popular because of it's simlicity and is accomplished through forming intelligent charge plans. These plans can take on multiple considerations including bus availability, environmental impact \cite{zhou_bi-objective_2021}, battery health \cite{houbbadi_optimal_2019}, and the cost of electricity.
\par electrical cost is based on the cost of generating power, and the cost of maintaining the distribution hardare. Large charge rates stress the existing infrastructure and therefore incur additional cost.
\par Methods to decrease energy expenses have been formed by decreasing the load exigence \cite{cheng_smart_2020}, \cite{ojer_development_2020}, \cite{qin_numerical_2016}, \cite{bagherinezhad_spatio-temporal_2020} or minimising cost directly.
\par Contributions: This work builds on \cite{brown_position_nodate} by encorporating the rate schedule from \cite{rocky_mountain_power_rocky_2021} as the linear objective function from \cite{mortensen_comprehensive_2021} which includes on-peak and off-peak rates, uncontrolled loads, and rate-sensative fees known as demand and facilities charges.  
\par The rest of this paper is organized as follows: Section \ref{sec:4_formulation} discuses the basic problem formulation, Section \ref{sec:5_battery} discuses linear constraints that govern the behavior and limitations of the sate of charge. Section \ref{sec:uncontrolled} discuses how to incorporate uncontrolled loads into the optimization framework. Section \ref{sec:objective} explaines how the objective function is formed, and section \ref{sec:results} discusses performance.
