\section{Introduction}
\par  Battery powered electric motors for buses have long been a desired alternative to the internal combustion engine \cite{Mahmoud2016}. The reduced maintenance \cite{poornesh_comparative_2020}, zero emissions \cite{kato_comparative_2013}, and access to renewable energy \cite{cheng_smart_2020} are but some of the benefits that have caused transit authorities to begin incorporating them into their bus fleets.  
\par Despite their benefits, transitioning to BEBs must address additional challenges, in particular, their extended refuel times. If a diesal or CNG bus runs low on fuel, refueling takes five to ten minutes.  An electric bus on the other hand may require several hours, causing the bus to fall behind schedule.
\par Maintaining a schedule while staying charged is one of the main challenges that BEBs face, and requires careful planning which must account for the battery discharge along routes, charge times, and a limited number of chargers. 
\par Charging while a bus is in motion, or dynamic charging, is one way to simplify a charge plan. There are a number of ways to do this including overhead \cite{csonka_optimization_2021} and inductive charging \cite{jeong_automatic_2018} \cite{balde_electric_2019}. An overhead charging scenario allows the bus to charge on overhead power lines while in motion while inductive charging relies on specialized hardware in the roads to inductively transfer energy when a bus passes overhead. Both methods remove the need to stop for service and allow an electrical vehicle to stay in service indefinately. They also require extensive infrastructure that may not be available.
\par In the absence of infrastructure, \cite{jain_battery_2020} and \cite{xian_zhang_optimal_2016} have proposed methods that exchange depleted batteries for fresh ones. Such a method would eliminate both the logistical challenges of planning and the infrastructure dependence of dynamic charging. The only drawback, is that BEBs are not built with battery exchanges in mind, therefore the task can require specialized hardware, technical expertise, or automation, all of which add complexity and cost.

\par One charge option that avoids both the infrastructural demands of dynamic charging and the technical difficulties of battery swapping is stationary charging, which plans rest periods into a bus's schedule during which that bus can charge \cite{whitaker_network_nodate}. Stationary charging is the least invasive form of bus charging because it only requires charging hardware at specific locations and makes no changes to bus batteries. Prior work in this area addresses a number of problems including distributed charging networks \cite{Nimalsiri2020}, bus availability, environmental impact \cite{zhou_bi-objective_2021}, route scheduling \cite{Rinalde_Mixed_2020}, battery health \cite{houbbadi_optimal_2019}, the cost of electricity \cite{Leou_optimal_2017} and the cost of charging infrastructure \cite{Wei2018}.
\par One drawback to using a stationary charging solution is that it does require significant rest periods for charging. One way to decrease the charge intervals is to use high power chargers, which deliver more energy in a smaller period of time. Large power demands however do increase the overall cost of energy because they must be supported by highly capable infrastructure \cite{stahleder_impact_2019}, \cite{deb_impact_2017}, \cite{boonraksa_impact_2019}. An effective charge plan must therefore balance the need to charge quickly with the desire to maintain a low power profile \cite{cheng_smart_2020}, \cite{ojer_development_2020}, \cite{qin_numerical_2016}, \cite{bagherinezhad_spatio-temporal_2020}, \cite{Wang2019} which includes power used by BEBs and the power needed by other consumers.  
\par Because the additional power users are outside the control of the charge plan, their power requirements are referred to in this paper as ``uncontrolled loads''. Uncontrolled loads complicate the problem of finding an optimal charge plan. If a bus charges in the presence of a large uncontrolled load, the overall power profile is heightened, increasing cost for energy. 
\par A significant contribution of \cite{mortensen_comprehensive_2021} was how the authors minimized the the financial impact of charging in the presence of uncontrolled loads by formulating the charge problem as a graph and solving for the optimal path. The graph based approach represented time discretely, which lent itself well to integrating uncontrolled loads, which are sampled discretely in practice. 
\par Unfortunately, more temporal precision decreases the time step between sections in the graph, which leads to a larger graph and significantly increases the computational complexity. The authors of \cite{brown_position_nodate} compute the charge schedule continuously by formulating the charge problem as a bin packing problem \cite{Ma_Mixed-integer_2017}, yielding a precise time schedule for charging, but finding a continuous-time schedule that incorporates uncontrolled loads and minimizes a comprehensive cost function remains an open problem and is the focus of this paper. 
\par The rest of this paper is organized as follows: Section \ref{sec:4_formulation} discuses the basic problem formulation, Section \ref{sec:5_battery} discuses linear constraints that govern the behavior and limitations of the sate of charge. Section \ref{sec:uncontrolled} discuses how to incorporate uncontrolled loads into the optimization framework. Section \ref{sec:objective} explaines how the objective function is formed, and Section \ref{sec:results} discusses performance.
