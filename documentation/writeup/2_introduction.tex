\section{Introduction}
\begin{enumerate}
	\item Electric buses are good because
		\begin{enumerate}
			\item They incur less maintenance \cite{poornesh_comparative_2020}(Poornesh et al.)
			\item They are eco-friendly \cite{kato_comparative_2013}
				\begin{enumerate}
					\item Provide access to renewable energy \cite{cheng_smart_2020}
					\item Give off zero emissions
				\end{enumerate}
		\end{enumerate}
	\item Electric buses struggle with
		\begin{enumerate}
			\item Extended charge times impact logistics (can take anyware from 100 minutes to 8 hours for a full charge)
			\item Extended charge times are reduced with high power chargers
			\item High charge rates taxe electrical infrastructure and increase the overall cost of energy. \cite{stahleder_impact_2019}, \cite{deb_impact_2017}, \cite{boonraksa_impact_2019}
		\end{enumerate}
	\item Managing electric bus fleet logistics can be handled on two levels
		\begin{enumerate}
			\item installation design: addresses questions such as what should the routes be, where should chargers be located, and what kinds of charging hardware do we use.
			\item charge scheduling: addresses the question of when should buses charge given the constraints from the installaion. 
		\end{enumerate}
	\item In most circumstances, hardare for charging is already installed or options are limited (with some exceptions (Ojer, et al)) and thus, the majority of work focuses on optimally utilizing a variety of charging hardware including
		\begin{enumerate}
			\item Dynamic overhead charging \cite{csonka_optimization_2021}
			\item Dynamic inductive charging \cite{jeong_automatic_2018} \cite{balde_electric_2019}
			\item battery swapping \cite{jain_battery_2020} \cite{xian_zhang_optimal_2016}
			\item stationary charging (A. Jahic),\cite{whitaker_network_nodate}
		\end{enumerate}
	\item each hardare type has its merits. Both the dynamic overhead and inductive charging eliminate the need for planning and down time as buses are charged while in service, but require expensive infrastructure that may not be feasible to install or purchase.
	\item battery swapping doesn't require any type of hardware installation and removes limitations imposed by route schedules, but requires specialized tools and/or automation.
	\item stationary charging is the least invasive form of bus charging because it only requires charging hardware at specific locations and makes no changes to bus batteries. 
	\item stationary charging is accomplished through forming intelligent charge plans. These plans can take on multiple considerations including bus availability, environmental impact \cite{zhou_bi-objective_2021}, battery health \cite{houbbadi_optimal_2019}, and the cost of electricity.
	\item electrical cost is based on the cost of generating power, and the cost of maintaining the distribution hardare. Large charge rates stress the existing infrastructure and therefore incur additional cost.
	\item Methods to decrease energy expenses have been formed by decreasing the load exigence \cite{cheng_smart_2020}, \cite{ojer_development_2020}, \cite{qin_numerical_2016}, \cite{bagherinezhad_spatio-temporal_2020} or minimising cost directly.
	\item Contributions: This work builds on \cite{brown_position_nodate} by encorporating the rate schedule from \cite{rocky_mountain_power_rocky_2021} as the linear objective function from \cite{mortensen_comprehensive_2020} which includes on-peak and off-peak rates, uncontrolled loads, and rate-sensative fees known as demand and facilities charges.  
	\item The rest of this paper is organized as follows: 
\end{enumerate}
