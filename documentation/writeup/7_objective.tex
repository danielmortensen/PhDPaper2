\section{Objective Function\label{sec:objective}}
This work adopts the objective function developed in \textcolor{red}{insert reference to our prior paper here}, which implements the rate schedule from \textcolor{red}{insert reference to rocky mountain power}. The rate schedule in \textcolor{red}{insert reference to rocky mountain power rate schedule} is based off of two primary components: power, and energy.  
\par Power is billed per kW for the highest 15 minute average power over a desired period of time. It is common practice for power providers to designate "on-peak" periods when power is generally in high demand, and "off-peak" hours for all other time periods. 
\par The rate schedule given in \textcolor{red}{insert reference to rocky mountain here} assesses a fee for a users maximum average power during on-peak hours called the demand charge, and a user's overall maximum average power, called a facilities charge as shown in figure \ref{fig:charges}. 
\par Energy fees are also assessed per kWh of energy consumed with a higher rate for energy consumed during on-peak hours and a lower rate for energy consumed during off-peak hours.
\begin{figure*}
	\centering
	\begin{tabular}{c | c c c}
		            & On-Peak               & Off-Peak               & Both \\ \hline
		Energy      & On-Peak Energy Charge & Off-Peak Energy Charge & None \\
		Energy Rate & $u_{\text{e-on}}$     & $u_{\text{e-off}}$     & None \\
		Power       & Demand Charge         & None                   & Facilities Charge \\
		Power Rate  & $u_{\text{p-on}}$     & None                   & $u_{\text{p-all}}$
	\end{tabular}
	\caption{Description of the assumed billing structure}
	\label{fig:charges}
\end{figure*}

\subsection{Power Charges}
It is necessary to compute the maximum power both overall and for on-peak periods. In section \ref{sec:uncontrolled}, $\Delta T$ was used to denote the time offset between power samples and that each power reading would reflect the average power used in the previous interval. 
\par In this section, $\Delta T$ is set to 15 minutes, making $\mathbf{p}_{\text{total}}$ an expression of the 15 minute average power. Next, let $\mathcal{S}_{\text{on}}$ be the set of all indices belonging to on-peak time periods such that $j\in \mathcal{S}_{\text{on}} \Rightarrow p_j^{\text{total}} $ represents a 15 minute average during an on-peak interval.  Additionally, let $q_{\text{on}}$ be the maximum on-peak average power.  The constriants for determining the maximum on-peak average are defined as
\begin{equation} \begin{aligned}
	p_j^{\text{total}} &\le q_{\text{on}} \ \forall j \in \mathcal{S}_{\text{on}} \\
	p_j^{\text{total}} - q_{\text{on}} &\le 0 \ \forall j \in \mathcal{S}_{\text{on}}\\
	\begin{bmatrix}0 & \hdots & 0 & 1_j & 0 & \hdots & 0 & -1_{\text{on}} & 0 \end{bmatrix}\mathbf{y} &\le 0 \forall j\in \mathcal{S}_{on} \\
		A_{\text{ineq9}} &\le \mathbf{0}.
\end{aligned} \end{equation}
Because an increased value in $q_{\text{on}}$ is directly related to an increase in cost, the optimizer will minimize $q_{\text{on}}$ until it is equal to the maximum value in $\{p_j^{\text{total}} \ \forall j \in \mathcal{S}_{\text{on}}\}$. A similar proceedure can be used to derive a set of constraints for the overall maximum average power, denoted $q_{\text{all}}$, and is represented as
\begin{equation} 
	A_{\text{ineq10}}\mathbf{y} \le  \mathbf{0}.
\end{equation}
The charges for power are then expressed as 
\begin{equation} \begin{aligned}
	\text{power cost} &= q_{\text{on}}\cdot u_{\text{p-on}} + q_{\text{all}} \cdot u_{\text{p-all}} \\
	                  &= \begin{bmatrix}u_{\text{p-on}} & 0 & \hdots & 0 & u_{\text{p-all}} \end{bmatrix}\mathbf{y} \\
		          &= \mathbf{u}_{\text{p}}^T\mathbf{y}
\end{aligned}\end{equation}
where $u_{\text{p-on}}$ is the rate per kW for on-peak power use, or the demand charge and $u_{\text{p-all}}$ is the rate per kW for the overall maximum 15 minute average.
\subsection{Energy Charges}
Energy is defined as the integral of power over a length of time.  Because the values for power given in this work reflect an average power, the energy over a given period can be computed by multiplying the average power by the change in time, or $\Delta T$ such that
\begin{equation}\begin{aligned}
	\text{Total Energy} = \mathbf{1}^T\mathbf{p}_{\text{total}}\cdot \Delta T.
\end{aligned}\end{equation}
However, because the energy is billed for on-peak and off-peak time periods, we define two binary vectors $\mathbf{1}_{\text{on}}$ and $\mathbf{1}_{\text{off}}$ such that $1^{\text{on}}_j = 1$ when $p_j^{\text{total}}$ represents an average power from an on-peak time period and $1_{\text{off}} = 1$ when $p_j^{\text{total}}$ represents the average power usage for an off-peak interval. The on-peak and off-peak energy can be computed as
\begin{equation}\begin{aligned}
	\text{On-Peak Energy} = \mathbf{1}_{\text{on}}^T\mathbf{p}_{\text{total}}\cdot\Delta T\\
	\text{Off-Peak Energy} = \mathbf{1}_{\text{off}}^T\mathbf{p}_{\text{total}}\cdot\Delta T.
\end{aligned}\end{equation}
Let $u_{\text{e-on}}$ and $u_{\text{e-off}}$ represent the on-peak and off-peak energy rates respectively. The total cost for energy is computed as
\begin{equation} \begin{aligned}
	\text{Energy Cost} &= \mathbf{1}_{\text{on}}^T\mathbf{p}_{\text{total}}\cdot u_{\text{e-on}}\cdot\Delta T + \mathbf{1}_{\text{off}}^T\mathbf{p}_{\text{total}}\cdot u_{\text{e-off}}\cdot\Delta T \\
			   &= \left(\mathbf{1}_{\text{on}}\cdot u_{\text{e-on}}\cdot\Delta T \right )^T\mathbf{p}_{\text{total}} + \left(\mathbf{1}_{\text{off}}\cdot u_{\text{e-off}}\cdot\Delta T \right )^T\mathbf{p}_{\text{total}} \\
			   &= \mathbf{u}_{\text{e-on}}^T\mathbf{p}_{\text{total}} + \mathbf{u}_{\text{e-off}}^T\mathbf{p}_{\text{total}} \\
			   &= \left(\mathbf{u}_{\text{e-on}} + \mathbf{u}_{\text{e-off}}\right )^T\mathbf{p}_{\text{total}}\\
			   &= \mathbf{\tilde{u}}_{\text{e}}^T\mathbf{p}_{\text{total}} \\
			   &= \mathbf{u}_{\text{e}}^T\mathbf{y}
\end{aligned} \end{equation}
\subsection{Cost Function and Final Problem}
The entire cost function is given as the sum of the energy and power costs such that
\begin{equation}\begin{aligned}
	\text{Cost} &= \mathbf{u}_{\text{p}}^T\mathbf{y} + \mathbf{u}_{\text{e}}^T\mathbf{y} \\
	            &= \left( \mathbf{u}_{\text{p}} + \mathbf{u}_{\text{e}} \right )^T\mathbf{y} \\
		    &= \mathbf{g}^T\mathbf{y}
\end{aligned}\end{equation}
The complete problem can now be formulated as
\begin{equation}\begin{matrix}
	\underset{\mathbf{y}}{\text{min}} \ \mathbf{y}^T\mathbf{g} \text{ subject to } \\
	\begin{bmatrix}
		A_{\text{eq1}} \\ 
		A_{\text{eq2}} \\
		A_{\text{eq3}} \\
		A_{\text{eq4}} \\
		A_{\text{eq5}} \\
		A_{\text{eq6}}
	\end{bmatrix}\mathbf{y} = 
	\begin{bmatrix}
		\mathbf{b}_{\text{eq1}} \\
		\mathbf{b}_{\text{eq2}} \\
		\mathbf{b}_{\text{eq3}} \\
		\mathbf{b}_{\text{eq4}} \\
		\mathbf{b}_{\text{eq5}} \\
		\mathbf{b}_{\text{eq6}} \\
	\end{bmatrix}, \ 
	 \begin{bmatrix}
		A_{\text{ineq1}} \\
		A_{\text{ineq2}} \\
		A_{\text{ineq3}} \\
		A_{\text{ineq4}} \\
		A_{\text{ineq5}} \\
		A_{\text{ineq6}} \\
		A_{\text{ineq7}} \\
		A_{\text{ineq8}} \\
		A_{\text{ineq9}} \\
		A_{\text{ineq10}}\\
	 \end{bmatrix}\mathbf{y} \le 
	 \begin{bmatrix}
		\mathbf{b}_{\text{ineq1}} \\
		\mathbf{b}_{\text{ineq2}} \\
		\mathbf{b}_{\text{ineq3}} \\
		\mathbf{b}_{\text{ineq4}} \\
		\mathbf{b}_{\text{ineq5}} \\
		\mathbf{b}_{\text{ineq6}} \\
		\mathbf{b}_{\text{ineq7}} \\
		\mathbf{b}_{\text{ineq8}} \\
		\mathbf{b}_{\text{ineq9}} \\
		\mathbf{b}_{\text{ineq10}}\\
	 \end{bmatrix}
\end{matrix} \end{equation}
or 
\begin{equation}\begin{matrix}
	\underset{\mathbf{y}}{\text{min}} \ \mathbf{y}^T\mathbf{g} \text{ subject to } \\
	A_{\text{eq}}\mathbf{y} = \mathbf{b}_{\text{eq}}, \ A_{\text{ineq}}\mathbf{y} \le \mathbf{b}_{\text{ineq}},
\end{matrix} \end{equation}

