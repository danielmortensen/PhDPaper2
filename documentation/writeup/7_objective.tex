\section{Objective Function\label{sec:objective}}
This work adopts the objective function developed in \cite{mortensen_comprehensive_2021}, which implements the rate schedule from \cite{rocky_mountain_power_rocky_2021}. The rate schedule in \cite{rocky_mountain_power_rocky_2021} is based off of two primary components: power, and energy.  
\par Power is billed per kW for the highest 15 minute average power over a fixed period of time. It is common practice for power providers to use a higher rate during ``on-peak'' periods when power is in higher demand and use a lower rate during ``off-peak'' hours, which account for all other time periods. 
\par The rate schedule given in \cite{rocky_mountain_power_rocky_2021} assesses a fee for a users maximum average power during on-peak hours called the On-Peak Power charge, and a user's overall maximum average power, called a facilities charge as shown in figure \ref{fig:charges}. 
\par Energy fees are also assessed per kWh of energy consumed with a higher rate for energy consumed during on-peak hours and a lower rate for energy consumed during off-peak hours.
\begin{figure*}
	\centering
	\begin{tabular}{c | c c c}
		Charge Type & On-Peak               & Off-Peak               & Both \\ \hline
		Energy      & On-Peak Energy Charge & Off-Peak Energy Charge & None \\
		Power       & Demand Charge         & None                   & Facilities Charge 
	\end{tabular}
	\caption{Description of the assumed billing structure}
	\label{fig:charges}
\end{figure*}

\subsection{Power Charges}
It is necessary to compute the maximum power both overall and for on-peak periods. Section \ref{sec:uncontrolled} adopted the convention that $\Delta T$ denotes the time offset between power samples and that each power reading would reflect the average power used in the previous interval. Now let us set $\Delta T$ to 15 minutes, making $\mathbf{p}_{\text{total}}$ an expression of the 15 minute average power. Next, let $\mathcal{S}_{\text{on}}$ be the set of all indices belonging to on-peak time periods such that $j\in \mathcal{S}_{\text{on}}$ implies that the $j^{\text{th}}$ element of $\mathbf{p}^{\text{total}}$, $p_j^{\text{total}}$, represents a 15 minute average during an on-peak interval and let $q_{\text{on}}$ be the maximum on-peak average power.  With these definitions, constriants for determining the maximum on-peak average are defined as
\begin{equation} \begin{aligned}
	p_j^{\text{total}} &\le q_{\text{on}} \ \forall j \in \mathcal{S}_{\text{on}} \\
	\begin{bmatrix} 1 & -1\end{bmatrix} \begin{bmatrix}p_j^{\text{total}} \\ q_{\text{on}} \end{bmatrix} &\le 0 \ \forall j \in \mathcal{S}_{\text{on}}\\
		A_{11}\mathbf{y} &\le \mathbf{0} \\
		A_{11}\mathbf{y} &\le \mathbf{b}_{11}
\end{aligned} \end{equation}
Because an increased value in $q_{\text{on}}$ is directly related to an increase in cost, the optimizer will minimize $q_{\text{on}}$ until it is equal to the maximum value in $\{p_j^{\text{total}} \ \forall j \in \mathcal{S}_{\text{on}}\}$. A similar proceedure can be used to derive a set of constraints for the overall maximum average power, denoted $q_{\text{all}}$, and is represented as
\begin{equation} \begin{aligned}
	A_{12}\mathbf{y} &\le  \mathbf{0} \\
	A_{12}\mathbf{y} &\le \mathbf{b}_{12}.
\end{aligned} \end{equation}
The charges for power are then expressed as 
\begin{equation} \begin{aligned}
	\text{power cost} &= q_{\text{on}}\cdot u_{\text{p-on}} + q_{\text{all}} \cdot u_{\text{p-all}} \\
			  &= \begin{bmatrix}u_{\text{p-on}} & u_{\text{p-all}} \end{bmatrix}\begin{bmatrix}q_{\text{on}} \\ q_{\text{all}} \end{bmatrix} \\
		          &= \mathbf{u}_{\text{p}}^T\mathbf{y}
\end{aligned}\end{equation}
where $u_{\text{p-on}}$ is the rate per kW for on-peak power use, or the demand charge and $u_{\text{p-all}}$ is the rate per kW for the overall maximum 15 minute average.
\subsection{Energy Charges}
Energy is defined as the integral of power over a length of time.  Because the values for power given in this work reflect an average power, the energy over a given period can be computed by multiplying the average power by the change in time, or $\Delta T$ such that
\begin{equation}\begin{aligned}
	\text{Total Energy} = \mathbf{1}^T\mathbf{p}_{\text{total}}\cdot \Delta T.
\end{aligned}\end{equation}
However, because the energy is billed for on-peak and off-peak time periods, we define two binary vectors $\mathbf{1}_{\text{on}}$ and $\mathbf{1}_{\text{off}}$ such that $1^{\text{on}}_j = 1$ $\forall j \in S_{\text{on}}$ and zero otherwise. Similarly,  $1_{\text{off}} = \mathbf{1} - 1_{\text{on}}$. The on-peak and off-peak energy can then be computed as
\begin{equation}\begin{aligned}
	\text{On-Peak Energy} = \mathbf{1}_{\text{on}}^T\mathbf{p}_{\text{total}}\cdot\Delta T\\
	\text{Off-Peak Energy} = \mathbf{1}_{\text{off}}^T\mathbf{p}_{\text{total}}\cdot\Delta T.
\end{aligned}\end{equation}
Let $u_{\text{e-on}}$ and $u_{\text{e-off}}$ represent the on-peak and off-peak energy rates respectively. The total cost for energy is computed as
\begin{equation} \begin{aligned}
	\text{Energy Cost} &= \left(\mathbf{1}_{\text{on}}\cdot u_{\text{e-on}}\cdot\Delta T \right )^T\mathbf{p}_{\text{total}} + \left(\mathbf{1}_{\text{off}}\cdot u_{\text{e-off}}\cdot\Delta T \right )^T\mathbf{p}_{\text{total}} \\
			   &= \left(\mathbf{u}_{\text{e-on}} + \mathbf{u}_{\text{e-off}}\right )^T\mathbf{p}_{\text{total}}\\
			   &= \mathbf{u}_{\text{e}}^T\mathbf{y}
\end{aligned} \end{equation}
\subsection{Cost Function and Final Problem}
The entire cost function is given as the sum of the energy and power costs such that
\begin{equation}\begin{aligned}
	\text{Cost} &= \mathbf{u}_{\text{p}}^T\mathbf{y} + \mathbf{u}_{\text{e}}^T\mathbf{y} \\
	            &= \left( \mathbf{u}_{\text{p}} + \mathbf{u}_{\text{e}} \right )^T\mathbf{y} \\
		    &= \mathbf{g}^T\mathbf{y}
\end{aligned}\end{equation}
The complete problem can now be formulated as
\begin{equation}\begin{matrix}
	\underset{\mathbf{y}}{\text{min}} \ \mathbf{y}^T\mathbf{g} \text{ subject to } \\
	\begin{bmatrix}
		\tilde{A}_{1} \\ 
		\tilde{A}_{2} \\
		\tilde{A}_{3} \\
	\end{bmatrix}\mathbf{y} = 
	\begin{bmatrix}
		\tilde{\mathbf{b}}_{1} \\
		\tilde{\mathbf{b}}_{2} \\
		\tilde{\mathbf{b}}_{3} \\
	\end{bmatrix}, \ 
	\begin{bmatrix}
		A_{1} \\
		A_{2} \\
		A_{3} \\
		A_{4} \\
		A_{5} \\
		A_{6} \\
		A_{7} \\
		A_{8} \\
		A_{9} \\
		A_{10}\\
		A_{11}\\
		A_{12}\\
	 \end{bmatrix}\mathbf{y} \le 
	 \begin{bmatrix}
		\mathbf{b}_{1} \\
		\mathbf{b}_{2} \\
		\mathbf{b}_{3} \\
		\mathbf{b}_{4} \\
		\mathbf{b}_{5} \\
		\mathbf{b}_{6} \\
		\mathbf{b}_{7} \\
		\mathbf{b}_{8} \\
		\mathbf{b}_{9} \\
		\mathbf{b}_{10}\\
		\mathbf{b}_{11}\\
		\mathbf{b}_{12}\\
	 \end{bmatrix}
\end{matrix} \end{equation}
or 
\begin{equation}\begin{matrix}
	\underset{\mathbf{y}}{\text{min}} \ \mathbf{y}^T\mathbf{g} \text{ subject to } \\
	\tilde{A}\mathbf{y} = \tilde{\mathbf{b}}, \ A\mathbf{y} \le \mathbf{b},
\end{matrix} \end{equation}

