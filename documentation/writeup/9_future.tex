\section{Conclusions and Future Work }
In conclusions, the proposed algorithm yields significant cost savings over the work given in \cite{He_2019_Fast} and the baseline by decreases the 15-minute average power both overall and during on-peak periods. Additionally, time comparisons show that the proposed method offers high temporal resolution at no additional computational cost wheras the computations in prior work require several orders of magnitude more for a one minute time resolution.
\par Note that this work offers a framework for computing a globally optimal charge plan, however computational constraints render it unuseable for real-time updates. Future work might include approximate solutions from simpler hieristic approaches to reduce the computational complexity.  Another approach could be to use additional methods that address real-time deviations by taking actions that best get back to the global plan.
\par Another known limitation includes how the computational complexity for the current method does not scale with large numbers of buses (more than 30). For larger bus fleets, a descentralized as opposed to a global method might work better. 
\par Finally, this method does not account for uncertainty in the model.  Stochastic events such as arrival times, deviations in uncontrolled loads, and battery discharge can significantly affect the useability of the global plan. There are techniques which account for uncertainty in similar frameworks which may be helpful in preventing unforeseen consequences.
