\section{Conclusions and Future Work }
In conclusion, the proposed method yields significant cost savings over the work given in \cite{He_2019_Fast}, and the default charging behavior at the UTA site. This is accomplished by minimizing on-peak energy, on-peak power, and overall average power in the presence of uncontrolled loads. 
\par Future work might integrate approximate solutions from simpler heuristic approaches to initialize the MILP solver to accelerate convergence. Another known limitation includes how the computational complexity for the current method does not scale with large numbers of buses (more than 30). For larger bus fleets, a decentralized as opposed to a global method might work better. 
\par Finally, this method does not account for uncertainty in the model.  Stochastic events such as random arrival times, deviations in actual uncontrolled loads relative to historic values, and uncertainty in battery discharge can significantly affect the usability of the global plan. Accounting for stochastic models of variability can add robustness to the charging solution.  

