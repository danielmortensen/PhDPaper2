\section{Integrating Uncontrolled Loads}
Next, we desire to integrate the charge decisions with uncontrolled loads.  These loads are sampled discretly and so the results for the power use must be converted into a compatible format.  To start, define
\begin{equation}
	\begin{aligned}
		k^{\text{start}}_{ij}\cdot\Delta T + r^{\text{start}}_{ij}&\ge c_{ij} \\
		k^{\text{start}}_{ij}\cdot\Delta T + r^{\text{start}}_{ij}&\le c_{ij} \\
		k^{\text{end}}_{ij}\cdot\Delta T + r^{\text{end}}_{ij}&\ge s_{ij} \\
		k^{\text{end}}_{ij}\cdot\Delta T + r^{\text{end}}_{ij}&\le s_{ij} \\
	k^{\text{start}}_{ij}, k^{\text{end}}_{ij} \in \mathbb{Z} \\
	0 < r^{\text{start}}_{ij}, r^{\text{stop}}_{ij} < &\Delta T.
	\end{aligned}
\end{equation}
We desire to find binary vectors $\mathbf{s}^{\text{up}}_{ij}$, $\mathbf{s}^{\text{on}}_{ij}$, and $\mathbf{s}^{\text{off}}_{ij}$ which act as selectors for times indices where bus $i$ connects to, charges, and disconnects from a charger during the $j^{\text{th}}$ stop respectively. The on-ramping charging, and off-ramping periods will be used to represent power use from $k^{\text{start}}_{ij}$ to $k^{\text{end}}_{ij} + 1$, $k^{\text{start}}_{ij} + 1$ to $k^{\text{end}}_{ij}$ and $k^{\text{end}}_{ij}$ to $k^{\text{end}}_{ij} + 1$ respectively. 
\par Let $\mathbf{f}$ be a vector of one-based integer indices such that $\mathbf{f}_w = w \ \forall w \in (1,\text{nTime})$. The values in $\mathbf{s}^{\text{up}}_{ij}$ can be defined by the constraint
\begin{equation}\begin{aligned}
	k^{\text{start}}_{ij} &= \mathbf{f}^T\mathbf{s}^{\text{up}}_{ij} \\
	1 &= \mathbf{1}^T\mathbf{s}^{\text{up}}_{ij} \\
	\mathbf{s}^{\text{up}}_{ij} &\in \{0,1\}.
\end{aligned} \end{equation}
Similarly, the values for $\mathbf{s}^{\text{off}}_{ij}$ can be defined as
\begin{equation} \begin{aligned}
	k^{\text{stop}}_{ij} &= \mathbf{f}^T\mathbf{s}^{\text{off}}_{ij}\\ 
	1 &= \mathbf{1}^T\mathbf{s}^{\text{off}}_{ij} \\
	\mathbf{s}^{\text{off}}_{ij} &\in \{0,1\}.
\end{aligned} \end{equation}
The final piece is to define $\mathbf{s}$. The set of constraints can be described as follows: 
\begin{equation} \begin{aligned}
	\mathbf{1}^T\mathbf{s}^{\text{on}}_{ij} &= k^{\text{end}}_{ij} - k^{\text{start}}_{ij} - 1 \\
	s_w\cdot f_w &\le k^{\text{end}}_{ij} + M(1 - s_w) \ \forall s_w \in \mathbf{s}^{\text{on}}_{ij}\\
	s_w\cdot f_w &\ge k^{\text{start}}_{ij} - M(1 - s_w) \ \forall s_w \in \mathbf{s}^{\text{on}}_{ij}\\ 
\end{aligned} \end{equation}
where $M$ is $2\cdot\text{nTime}$.
\par We next define the average per use for each charger during an interval. The on-ramp intervals can be constrained as
\begin{align}
	p^{\text{on-ramp}}_{ij} &= \frac{\text{power}\cdot (\Delta T - r^{\text{start}}_{ij})}{\Delta T}\\
	p^{\text{off-ramp}}_{ij} &= \frac{\text{power}\cdot r^{\text{stop}}_{ij}}{\Delta T}\\
	p &= \text{power},
\end{align}
where $p_{ij}^{\text{on-ramp}}$, $p_{ij}^{\text{off-ramp}}$, and $p$ represent the average power for the on-ramping, off-ramping, and charging intervals respectively. The total average power use is calculated as 
\begin{align}\label{eqn:totalPower}
	\mathbf{p}_{\text{total}} = \mathbf{p} + \sum_{ij} \mathbf{s}^{\text{up}}_{ij}\cdot p^{\text{on-ramp}}_{ij} + \mathbf{s}^{\text{on}}_{ij}\cdot p + \mathbf{s}^{\text{off}}_{ij}\cdot p^{\text{off-ramp}}_{ij}
\end{align}
where $\mathbf{p}$ is the average power of the uncontrolled loads.
\par Note, however that the results from equation \ref{eqn:totalPower} contain a bilinear form. The first bilnear expression in  equation \ref{eqn:totalPower} can be rewritten as 
\begin{equation}
	\begin{aligned}
		p_w &\ge p^{\text{on-ramp}}_{ij} - M(1 - s_w) \ \forall s_w \in \mathbf{s}^{\text{up}}_{ij}, \ p_w \in \mathbf{p}_{ij}^{\text{up}}\\
		p_w &\le p^{\text{on-ramp}}_{ij} + M(1 - s_w) \ \forall s_w \in \mathbf{s}^{\text{up}}_{ij}, \ p_w \in \mathbf{p}_{ij}^{\text{up}}\\
		p_w &\ge -Ms_w \ \forall s_w \in \mathbf{s}^{\text{up}}_{ij}, \ p_w \in \mathbf{p}_{ij}^{\text{up}}\\
		p_w &\le Ms_w \ \forall s_w \in \mathbf{s}^{\text{up}}_{ij}, \ p_w \in \mathbf{p}_{ij}^{\text{up}}
	\end{aligned}
\end{equation}
and similarly the second as, 
\begin{equation} \begin{aligned}
	p_w &\ge p^{\text{off-ramp}}_{ij} - M(1 - s_w) \ \forall s_w \in \mathbf{s}^{\text{off}}_{ij}, \ p_w \in \mathbf{p}_{ij}^{\text{off}}\\
		p_w &\le p^{\text{off-ramp}}_{ij} + M(1 - s_w) \ \forall s_w \in \mathbf{s}^{\text{off}}_{ij}, \ p_w \in \mathbf{p}_{ij}^{\text{off}}\\
		p_w &\ge -Ms_w \ \forall s_w \in \mathbf{s}^{\text{off}}_{ij}, \ p_w \in \mathbf{p}_{ij}^{\text{off}}\\
		p_w &\le Ms_w \ \forall s_w \in \mathbf{s}^{\text{off}}_{ij}, \ p_w \in \mathbf{p}_{ij}^{\text{off}}\\
\end{aligned} \end{equation} 

An expression for the total power used can then be derived as
\begin{equation}
	\begin{aligned}
		\mathbf{p}_{\text{total}} = \mathbf{p} + \sum_{ij} \mathbf{p}^{\text{up}}_{ij} + \mathbf{p}^{\text{off}}_{ij} + \mathbf{s}^{\text{on}}_{ij}\cdot p
	\end{aligned}
\end{equation}
TODO: 
\begin{enumerate}
	\item Figure out notation for each bus and how to integrate it into this framework
	\item flush out constraints and get into matrix/standard form
\end{enumerate}
