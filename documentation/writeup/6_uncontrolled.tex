\section{Integrating Uncontrolled Loads\label{sec:uncontrolled}}
A monthly power bill is made up of several charges, two of which depend on the maximum energy consumed over 15 minutes. This 15-minute average power includes energy that is consumed by loads other than bus chargers, or "uncontrolled loads". In practice, data for uncontrolled loads is sampled and therefore discrete. The representations for how buses use power in Section \ref{sec:5_battery} are continuous, making their effects difficult to integrate with a discrete uncontrolled load. This section integrates these uncontrolled loads into the planning framework by converting the continuous start and end points, $c_{ij}$ and $s_{ij}$ from section \ref{sec:4_formulation}, to a vector $\mathbf{p}_{ij}$, where the $n^{\text{th}}$ element of $\mathbf{p}_{ij}$ represents the average power over the interval from $t_{i-1}$ to $t_i$ from bus $i$ during route $j$. These route power vectors can be added together to form a discrete profile for the buses.
\par The first step is to separate each value for $c_{ij}$ and $s_{ij}$ into an integer-remainder pair such that
\begin{equation} \label{eqn:discrete1} \begin{aligned}
	\left ( k^{\text{start}}_{ij} - 1 \right ) \cdot\Delta T + r^{\text{start}}_{ij}&= c_{ij} \\
	\left (	k^{\text{end}}_{ij} - 1 \right ) \cdot\Delta T + r^{\text{end}}_{ij}&= s_{ij} \\
	k^{\text{start}}_{ij}, k^{\text{end}}_{ij} \in \mathbb{Z} \\
	0 < r^{\text{start}}_{ij}, r^{\text{end}}_{ij} < &\Delta T.
\end{aligned} \end{equation} 
where $k_{ij}^{\text{start}}$ and $k_{ij}^{\text{end}}$ represent the number of complete time intervals contained in $c_{ij}$ and $s_{ij}$, $r_{ij}^{\text{start}}$ and $r_{ij}^{\text{end}}$ give the remainders, and $\Delta T$ is the time difference between the desired discrete samples. Equation \ref{eqn:discrete1} can be rewriten in standard form and zero padded such that
\begin{equation} \begin{aligned}
	\begin{bmatrix}\Delta T &  1 & -1 & 0        & 0 &  0\\ 
		       0        &  0 & 0  & \Delta T & 1 & -1\\
	\end{bmatrix} 
	\begin{bmatrix} k_{ij}^{\text{start}} \\
		        r_{ij}^{\text{start}} \\
			c_{ij}                \\
			k_{ij}^{\text{end}}   \\
			r_{ij}^{\text{end}}   \\
			s_{ij}
	\end{bmatrix} &= 
	\begin{bmatrix} 0 \\
	                0 \\
	\end{bmatrix} \ \forall i,j \\ 
	\tilde{A}_2\mathbf{y} &= \tilde{\mathbf{b}}_2.
\end{aligned} \end{equation}
and 
\begin{equation} \begin{aligned}
	\begin{bmatrix} 0 & -1 & 0 & 0 &  0 & 0\\
			0 &  1 & 0 & 0 &  0 & 0\\
		        0 &  0 & 0 & 0 & -1 & 0\\
			0 &  0 & 0 & 0 &  1 & 0\\
	\end{bmatrix} 
	\begin{bmatrix} k_{ij}^{\text{start}} \\
		        r_{ij}^{\text{start}} \\
			c_{ij}                \\
			k_{ij}^{\text{end}}   \\
			r_{ij}^{\text{end}}   \\
			s_{ij}
	\end{bmatrix} &\le 
	\begin{bmatrix}  0\\
			\Delta T \\
			0 \\
			\Delta T
	\end{bmatrix} \ \forall i,j \\ 
	A_6\mathbf{y} &\le \mathbf{b}_6.  
\end{aligned} \end{equation}
The next step is to use $k$ and $r$ to compute three sets of binary vectors, denoted $\mathbf{g}^{\text{start}}_{ij}$, $\mathbf{g}^{\text{on}}_{ij}$, and $\mathbf{g}^{\text{end}}_{ij}$, which act as selectors for indices corresponding to charge times. The selectors will be used to form the power vectors $\mathbf{p}_{ij}$ for each bus at each route. The values in $\mathbf{g}_{ij}^{\text{start}}$ and $\mathbf{g}_{ij}^{\text{end}}$ are equal to 1 during intervals that contains energy from the remainders $r_{ij}^{\text{start}}$ and $r_{ij}^{\text{end}}$ and $\mathbf{g}_{ij}^{\text{on}}$ is equal to 1 for all time indices that buses charges the entire time. 
\par Let $\mathbf{f}$ be a vector of one-based integer indices such that $f_w = w \ \forall w \in (1,\text{nPoint})$, where nPoint is the desired number of discrete samples. The values in $\mathbf{g}^{\text{start}}_{ij}$ and $\mathbf{g}_{ij}^{\text{end}}$ are defined as
\begin{equation}\label{eqn:idxStart}\begin{aligned}
	k^{\text{start}}_{ij}        &= \mathbf{f}^T\mathbf{g}^{\text{start}}_{ij} \\
	k^{\text{end}}_{ij}         &= \mathbf{f}^T\mathbf{g}^{\text{end}}_{ij}  \\ 
	1                            &= \mathbf{1}^T\mathbf{g}^{\text{start}}_{ij} \\
	1                            &= \mathbf{1}^T\mathbf{g}^{\text{end}}_{ij}  \\
	\mathbf{g}^{\text{start}}_{ij}  &\in \{0,1\}^{\text{nPoint}}                \\
	\mathbf{g}^{\text{end}}_{ij} &\in \{0,1\}^{\text{nPoint}}.
\end{aligned} \end{equation}

Equation \ref{eqn:idxStart} can be expressed in standard form and zero padded to form additional constraints in terms of $\mathbf{y}$.
\begin{equation} \begin{aligned}
	\begin{bmatrix} 0 & \mathbf{0}^T & -1 & \mathbf{f}^T \\
		        0 & \mathbf{1}^T &  0 & 0            \\
		       -1 & \mathbf{f}^T & 0 & \mathbf{0}^T  \\
		        0 & 0            & 0 & \mathbf{1}^T 
	\end{bmatrix} 
	\begin{bmatrix} k_{ij}^{\text{start}}       \\
		        \mathbf{g}_{ij}^{\text{start}} \\ 
			k_{ij}^{\text{end}}        \\ 
			\mathbf{g}_{ij}^{\text{end}} 
	\end{bmatrix} &= 
	\begin{bmatrix} 0 \\ 
			1 \\
	                0 \\
			1
	\end{bmatrix} \ \forall i,j \\
	\tilde{A}_3\mathbf{y} &= \tilde{\mathbf{b}}_3.
\end{aligned} \end{equation}
The final piece is to define $\mathbf{g}^{\text{on}}$. The values in $\mathbf{g}^{\text{on}}$ must be both greater than $k_{ij}^{\text{start}}$, and less than $k_{ij}^{\text{end}}$ when they are not zero such that 
\begin{equation}\label{eqn:gOnCases}\begin{aligned}
	\begin{rcases}
		\begin{array}{ll}
			g_w f_w \le k^{\text{end}}  \\
			g_w f_w \ge k^{\text{start}}\\ 
		\end{array} & g_w = 1 \\
	\end{rcases}.
\end{aligned} \end{equation}
Equation \ref{eqn:gOnCases} can be expressed as a set of linear constraints such that
\begin{equation} \label{eqn:gOnBigM}\begin{aligned}
	g_w\cdot f_w &\le k^{\text{end}}_{ij} + M(1 - g_w) \\
	g_w\cdot f_w &\ge k^{\text{start}}_{ij} - M(1 - g_w) \\ 
\end{aligned} \end{equation}
where $M$ is $2\cdot\text{nPoint}$. The constraints in equation \ref{eqn:gOnBigM} do not require that all values between $k_{ij}^{\text{start}}$ and $k_{ij}^{\text{end}}$ be set to one however, only that if a value is equal to one, that it must be between $k_{ij}^{\text{start}}$ and $k_{ij}^{\text{end}}$. For all values between  $k_{ij}^{\text{start}}$ and $k_{ij}^{\text{end}}$ to be $1$, the sum of $\mathbf{g}_{ij}^{\text{on}}$ must be equal to the difference between $k_{ij}^{\text{end}}$ and $k_{ij}^{\text{start}}$ such that 
\begin{equation} \label{eqn:gOnSemiFinal}\begin{aligned}
	g_w\cdot f_w &\le k^{\text{end}}_{ij} + M(1 - g_w) \\
	g_w\cdot f_w &\ge k^{\text{start}}_{ij} - M(1 - g_w) \\ 
	\mathbf{1}^T\mathbf{g}_{ij}^{\text{on}} &= k_{ij}^{\text{end}} - k_{ij}^{\text{start}} - 1.\\
\end{aligned} \end{equation}
The constraints in equation \ref{eqn:gOnSemiFinal} work well for a general use case, however when $k_{ij}^{\text{end}}$ is equal to $k_{ij}^{\text{start}}$, the last constraint in equation \ref{eqn:gOnSemiFinal} becomes
\begin{equation}
	\mathbf{1}^T\mathbf{g}_{ij}^{\text{on}} = -1
\end{equation}
which leads to an empty feasible set because all the elements of $\mathbf{g}_{ij}^{\text{on}}$ are binary. Let $k_{ij}^{\text{eq}}$ be a binary variable which is equal to $0$ when $k_{ij}^{\text{end}}$ is not equal to $k_{ij}^{\text{start}}$. Equation \ref{eqn:gOnSemiFinal} can be modified to handle the case where $k_{ij}^{\text{end}}$ is equal to $k_{ij}^{\text{start}}$ as
\begin{equation} \label{eqn:gOnFinal}\begin{aligned}
	g_w\cdot f_w &\le k^{\text{end}}_{ij} + M(1 - g_w) \\
	g_w\cdot f_w &\ge k^{\text{start}}_{ij} - M(1 - g_w) \\ 
	\mathbf{1}^T\mathbf{g}_{ij}^{\text{on}} &= k_{ij}^{\text{end}} - k_{ij}^{\text{start}} - k_{ij}^{\text{eq}}.\\
\end{aligned} \end{equation} 
 The variable $k_{ij}^{\text{eq}}$ is defined as
\begin{equation}\label{eqn:kEq}\begin{aligned}
	k_{ij}^{\text{end}} - k_{ij}^{\text{start}} - M k_{ij}^{\text{eq}} \le 0 \\
	-k_{ij}^{\text{end}} + k_{ij}^{\text{start}} + M k_{ij}^{\text{eq}} \le M .
\end{aligned}\end{equation}
The constraints from equations \ref{eqn:gOnFinal} and \ref{eqn:kEq} can be expressed in standard form as
\begin{equation} \label{eqn:gOnFinalStd}\begin{aligned}
	\mathbf{1}^T\mathbf{g}_{ij}^{\text{on}} - k_{ij}^{\text{end}} + k_{ij}^{\text{start}} + k_{ij}^{\text{eq}} &=  0 \\
	k_{ij}^{\text{end}} - k_{ij}^{\text{start}} - M k_{ij}^{\text{eq}} &\le 0 \\
	-k_{ij}^{\text{end}} + k_{ij}^{\text{start}} + M k_{ij}^{\text{eq}} &\le M \\
	g_w\left (f_w + M \right) - k_{ij}^{\text{end}} &\le M \\
	g_w\left (M - f_w\right) + k_{ij}^{\text{start}} &\le M.
\end{aligned}\end{equation}
The inequality constriants from equation \ref{eqn:gOnFinalStd} imply that
\begin{equation} \label{eqn:gOnFinalPart1}\begin{aligned}
	\begin{bmatrix}f_w + M & -1 & 0\\
		       M - f_w & 0 & 1 
	\end{bmatrix} 
	\begin{bmatrix}g_w                 \\
		       k_{ij}^{\text{end}} \\ 
		       k_{ij}^{\text{start}}
	\end{bmatrix} \le
	\begin{bmatrix} M \\
	                M 
	\end{bmatrix} \forall g_w \in \mathbf{g}_{ij}^{\text{on}}.
\end{aligned}\end{equation} 
and that
\begin{equation} \label{eqn:kEqStd}\begin{aligned}
	\begin{bmatrix}1 & -1 & -M \\
		       -1 & 1 & M  \\
		       \end{bmatrix} \begin{bmatrix}k_{ij}^{\text{end}} \\ k_{ij}^{\text{start}} \\ k_{ij}^{\text{eq}} \end{bmatrix} \le \begin{bmatrix} 0 \\ M\end{bmatrix} \ \forall i,j
\end{aligned} \end{equation}
which can be concatenated for all $i,j$ and zero padded to form a joint matrix which has the form 
\begin{equation}
	A_7\mathbf{y} \le \mathbf{b}_7.
\end{equation}
Similarly, the equality constraint from equation \ref{eqn:gOnFinalStd} can also be concatenated and zero padded such that
\begin{equation} \begin{aligned}
	\mathbf{1}^T\mathbf{s}_{ij}^{\text{on}} - k_{ij}^{\text{end}} + k_{ij}^{\text{start}} + k_{ij}^{\text{eq}} &= 0 \ \forall i,j\\
	\begin{bmatrix}\mathbf{1}^T & - 1 & 1 & -1\end{bmatrix} \begin{bmatrix}\mathbf{g}_{ij}^{\text{on}} \\ k_{ij}^{\text{end}} \\ k_{ij}^{\text{start}} \\ k_{ij}^{\text{eq}} \end{bmatrix} &= 0 \\
		\tilde{A}_4\mathbf{y} &= \tilde{\mathbf{b}}_4.
\end{aligned} \end{equation}
	\par The next step is to define the average power during intervals that only charge for part of the time.  These intervals correspond to the remainder values $r_{ij}^{\text{start}}$ and $r_{ij}^{\text{end}}$ and, as with previous constraints, distinguish between behavior for $k_{ij}^{\text{eq}} = 0$ and $k_{ij}^{\text{eq}} = 1$.. The average power for $\mathbf{p}_{ij}$ that corresponds to $r_{ij}^{\text{start}}$ and $r_{ij}^{\text{end}}$ can be computed as 
\begin{equation}\label{eqn:avgPower1}\begin{aligned}
	\begin{rcases}
	\begin{array}{l} \begin{aligned}
		p^{\text{start}}_{ij} &= \frac{p\cdot (\Delta T - r^{\text{start}}_{ij})}{\Delta T}\\ 
		p^{\text{end}}_{ij} &= \frac{p\cdot r^{\text{end}}_{ij}}{\Delta T}.\\
	\end{aligned} \end{array} & k_{ij}^{\text{eq}} = 1 \\[8pt] 
	\begin{array}{l} \begin{aligned}
		p_{ij}^{\text{start}} &= \frac{p\cdot \left ( r_{ij}^{\text{end}} - r_{ij}^{\text{start}} \right )}{\Delta T} \\
		p_{ij}^{\text{end}} &= 0 \\
	\end{aligned} \end{array} & k_{ij}^{\text{eq}} = 0 \\
	\end{rcases}.
\end{aligned}\end{equation}
Equation \ref{eqn:avgPower1} can also be expressed as a set of linear inequality constraints such that
\begin{equation} \begin{aligned}
	p_{ij}^{\text{start}} &\le p - \frac{p}{\Delta T}r_{ij}^{\text{start}} + M\left ( 1 - k_{ij}^{\text{eq}} \right ) \\
	p_{ij}^{\text{start}} &\ge p - \frac{p}{\Delta T}r_{ij}^{\text{start}} - M\left ( 1 - k_{ij}^{\text{eq}} \right ) \\ 
	p_{ij}^{\text{start}} &\le \frac{p}{\Delta T}r_{ij}^{\text{end}} - \frac{p}{\Delta T}r_{ij}^{\text{start}} + Mk_{ij}^{\text{eq}}\\
	p_{ij}^{\text{start}} &\ge \frac{p}{\Delta T}r_{ij}^{\text{end}} - \frac{p}{\Delta T}r_{ij}^{\text{start}} - Mk_{ij}^{\text{eq}}\\ 
	p_{ij}^{\text{end}}   &\le \frac{p}{\Delta T}r_{ij}^{\text{end}} + M\left ( 1 - k_{ij}^{\text{eq}} \right )\\
	p_{ij}^{\text{end}}   &\ge \frac{p}{\Delta T}r_{ij}^{\text{end}} - M\left ( 1 - k_{ij}^{\text{eq}} \right )\\
	p_{ij}^{\text{end}}   &\le Mk_{ij}^{\text{eq}}\\
	p_{ij}^{\text{end}}   &\ge - Mk_{ij}^{\text{eq}}\\
\end{aligned} \end{equation}
where $M$ is the battery capacity and can be expressed in standard form as
\begin{equation} \begin{aligned}
	 p_{ij}^{\text{start}} + \frac{p}{\Delta T} r_{ij}^{\text{start}} + Mk_{ij}^{\text{eq}} &\le M + p \\
	-p_{ij}^{\text{start}} - \frac{p}{\Delta T} r_{ij}^{\text{start}} + Mk_{ij}^{\text{eq}} &\le M - p \\
	 p_{ij}^{\text{start}} - \frac{p}{\Delta T} r_{ij}^{\text{end}}  + \frac{p}{\Delta T}r_{ij}^{\text{start}} - Mk_{ij}^{\text{eq}} &\le 0\\
	-p_{ij}^{\text{start}} + \frac{p}{\Delta T} r_{ij}^{\text{end}}  - \frac{p}{\Delta T}r_{ij}^{\text{start}} - Mk_{ij}^{\text{eq}} &\le 0\\
	 p_{ij}^{\text{end}}   - \frac{p}{\Delta T} r_{ij}^{\text{end}} + Mk_{ij}^{\text{eq}} &\le M \\
	-p_{ij}^{\text{end}}   + \frac{p}{\Delta T} r_{ij}^{\text{end}} + Mk_{ij}^{\text{eq}} &\le M \\
	 p_{ij}^{\text{end}}   - Mk_{ij}^{\text{eq}} &\le 0\\
	-p_{ij}^{\text{end}}   - Mk_{ij}^{\text{eq}}&\le 0 \\
\end{aligned} \end{equation}
and finally as
\begin{equation}\begin{aligned}
	\begin{bmatrix} 
		 1 &  0 &  \frac{p}{\Delta T} &  0                  &  M\\
		-1 &  0 & -\frac{p}{\Delta T} &  0                  &  M\\
		 1 &  0 &  \frac{p}{\Delta T} & -\frac{p}{\Delta T} & -M\\ 
		-1 &  0 & -\frac{p}{\Delta T} &  \frac{p}{\Delta T} & -M\\
		 0 &  1 & 0                   & -\frac{p}{\Delta T} &  M\\
		 0 & -1 & 0                   &  \frac{p}{\Delta T} &  M\\
		 0 &  1 & 0                   &  0                  & -M\\
		 0 & -1 & 0                   &  0                  & -M\\
	\end{bmatrix} 
	\begin{bmatrix}
		p_{ij}^{\text{start}} \\
                p_{ij}^{\text{end}} \\
                r_{ij}^{\text{start}} \\
                r_{ij}^{\text{end}} \\
                k_{ij}^{\text{eq}}
	\end{bmatrix} &\le 
	\begin{bmatrix}
		M + p \\
		M - p \\
		0 \\
		0 \\
		M \\
		M \\
		0 \\
		0 \\
	\end{bmatrix} \ \forall i,j \\
	A_8 &\le \mathbf{b}_8
\end{aligned}\end{equation} 
where $p_{ij}^{\text{start}}$, $p_{ij}^{\text{end}}$, and $p$ represent the average power that corresponds to $r_{ij}^{\text{start}}$, $r_{ij}^{\text{end}}$, and full charging intervals respectively. The total average power use is calculated as 
\begin{align}\label{eqn:totalPower}
	\mathbf{p}_{\text{total}} = \bar{\mathbf{p}}_{\text{load}} + \sum_{ij} \mathbf{g}^{\text{start}}_{ij}\cdot p^{\text{start}}_{ij} + \mathbf{g}^{\text{on}}_{ij}\cdot p + \mathbf{g}^{\text{end}}_{ij}\cdot p^{\text{end}}_{ij}
\end{align}
where $\bar{\mathbf{p}}_{\text{load}}$ is the average power of the uncontrolled loads.
\par Note, however that the results from equation \ref{eqn:totalPower} contain a bilinear term. The first bilnear expression in  equation \ref{eqn:totalPower} must be rewritten as a vector containing values for $p_{ij}^{\text{start}}$ whenever $g_{ij}^{\text{start}} \ne 0$.  The resulting vector is denoted $\mathbf{p}_{ij}^{\text{start}}$ such that
\begin{equation} \label{eqn:bilinear1}\begin{aligned}
	\begin{rcases}
		p_w = p^{\text{start}} & g = 1 \\
		p_w = 0                & g = 0 \\
	\end{rcases} \forall p_w \in \mathbf{p}_{ij}^{\text{start}}
\end{aligned} \end{equation}
The constraints for equation \ref{eqn:bilinear1} can be rewritten as a set of linear inequality constraints such that
\begin{equation} \label{eqn:onPower}\begin{aligned}
	p_w &\ge p^{\text{start}}_{ij} - M(1 - g_w) \forall p_w \in \mathbf{p}_{ij}^{\text{start}}\\
	p_w &\le p^{\text{start}}_{ij} + M(1 - g_w) \forall p_w \in \mathbf{p}_{ij}^{\text{start}}\\
	p_w &\ge -Mg_w \forall p_w \in \mathbf{p}_{ij}^{\text{start}}\\
	p_w &\le  Mg_w \forall p_w \in \mathbf{p}_{ij}^{\text{start}}
\end{aligned} \end{equation}
The same approach can be taken to replace the other bilinear form $\mathbf{g}_{ij}^{\text{end}}\cdot p_{ij}^{\text{end}}$ with the vector $\mathbf{p}_{ij}^{\text{end}}$ as 
\begin{equation} \label{eqn:offPower} \begin{aligned}
	p_w &\ge p^{\text{end}}_{ij} - M(1 - g_w) \ \forall p_w \in \mathbf{p}_{ij}^{\text{end}}\\
	p_w &\le p^{\text{end}}_{ij} + M(1 - g_w) \ \forall p_w \in \mathbf{p}_{ij}^{\text{end}}\\
	p_w &\ge -Mg_w \ \forall p_w \in \mathbf{p}_{ij}^{\text{end}}\\
	p_w &\le Mg_w \  \forall p_w \in \mathbf{p}_{ij}^{\text{end}}.
\end{aligned} \end{equation} 
Equation \ref{eqn:onPower} can be written in standard form, stacked to accomodate the constraints for all $i,j$, and zero padded appropriately as
\begin{equation}\begin{aligned} 
	\begin{bmatrix}
		-1 & 1 & M \\
		1  & -1 & M \\
		-1 & 0 & -M \\
		1 & 0 & -M 
	\end{bmatrix}
	\begin{bmatrix} 
		p_w                     \\
	        p_{ij}^{\text{start}} \\
		g_w
	\end{bmatrix}  &\le
	\begin{bmatrix}
		M \\
		M \\
		0 \\
		0
	\end{bmatrix} \forall p_w \in \mathbf{p}_{ij}^{\text{start}}.\\ 
	A_9 &\le \mathbf{b}_9 
\end{aligned}\end{equation}
	Equation \ref{eqn:offPower} can be expressed in standard form, stacked for all $i,j$, and zero padded in a similar fashion such that 
\begin{equation}\begin{aligned} 
	\begin{bmatrix}
		-1 & 1 & M  \\
		1  & -1 & M \\
		-1 & 0 & -M \\
		1 & 0 & -M 
	\end{bmatrix}	
	\begin{bmatrix} p_w                      \\
		        p_{ij}^{\text{end}} \\
			g_w
	\end{bmatrix} &\le
	\begin{bmatrix} M \\
	                M \\
	                0 \\
	                0
	\end{bmatrix} \ \forall p_w \in \mathbf{p}_{ij}^{\text{end}} \\
	A_{10}\mathbf{y} & \le \mathbf{b}_{10}.
\end{aligned}\end{equation} 
An expression for the total power used can then be expressed as
\begin{equation}
	\begin{aligned}
		\mathbf{p}^{\text{total}} = \mathbf{p}^{\text{load}} + \sum_{ij} \mathbf{p}^{\text{start}}_{ij} + \mathbf{p}^{\text{end}}_{ij} + \mathbf{g}^{\text{on}}_{ij}\cdot p
	\end{aligned}
\end{equation}
and in standard form as
\begin{equation} \begin{aligned}
	\begin{bmatrix}
		1 & -1^{\text{start}} & -1^{\text{end}} & -1^{\text{on}}\cdot p  
	\end{bmatrix}
	\begin{bmatrix}
		\mathbf{p}_{w}^{\text{total}} \\
	        \mathbf{p}_{w}^{\text{start}} \\
	        \mathbf{p}_{w}^{\text{end}}   \\
	        \mathbf{g}_{w}^{\text{on}}    \\
	\end{bmatrix} &= \mathbf{p}_{w}^{\text{load}}  \\
	\tilde{A}_4\mathbf{y} &= \tilde{\mathbf{b}}_4
\end{aligned} \end{equation}

