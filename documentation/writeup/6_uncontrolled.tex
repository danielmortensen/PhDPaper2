\section{Integrating Uncontrolled Loads}
Next, we desire to integrate the charge decisions with uncontrolled loads.  These loads are sampled discretly and so the results for the power use must be converted into a compatible format.  To start, define
\begin{equation} \begin{aligned}
		k^{\text{start}}_{ij}\cdot\Delta T + r^{\text{start}}_{ij}&= c_{ij} \\
		k^{\text{end}}_{ij}\cdot\Delta T + r^{\text{end}}_{ij}&= s_{ij} \\
	k^{\text{start}}_{ij}, k^{\text{end}}_{ij} \in \mathbb{Z} \\
	0 < r^{\text{start}}_{ij}, r^{\text{stop}}_{ij} < &\Delta T.
\end{aligned} \end{equation} 
which can be expressed in standard form and zero padded such that
\begin{equation} \begin{aligned}
	\begin{bmatrix}\Delta T &  1 & -1 & 0        & 0 &  0\\ 
		       0        &  0 & 0  & \Delta T & 1 & -1\\
	\end{bmatrix} 
	\begin{bmatrix} k_{ij}^{\text{start}} \\
		        r_{ij}^{\text{start}} \\
			c_{ij}                \\
			k_{ij}^{\text{end}}   \\
			r_{ij}^{\text{end}}   \\
			s_{ij}
	\end{bmatrix} &= 
	\begin{bmatrix} 0 \\
	                0 \\
	\end{bmatrix} \ \forall i,j \\ 
	A_{\text{eq2}}\mathbf{y} &= \mathbf{b}_{\text{eq2}}.
\end{aligned} \end{equation}
and 
\begin{equation} \begin{aligned}
	\begin{bmatrix} 0 & -1 & 0 & 0 & 0 & 0\\
		        0 &  0 & 0 & 0 & 1 & 0\\
	\end{bmatrix} 
	\begin{bmatrix} k_{ij}^{\text{start}} \\
		        r_{ij}^{\text{start}} \\
			c_{ij}                \\
			k_{ij}^{\text{end}}   \\
			r_{ij}^{\text{end}}   \\
			s_{ij}
	\end{bmatrix} &\le 
	\begin{bmatrix} -\Delta T \\
			\Delta T
	\end{bmatrix} \ \forall i,j \\ 
	A_{\text{ineq5}}\mathbf{y} &\le \mathbf{b}_{\text{ineq5}}.  
\end{aligned} \end{equation}
We desire to find binary vectors $\mathbf{s}^{\text{up}}_{ij}$, $\mathbf{s}^{\text{on}}_{ij}$, and $\mathbf{s}^{\text{off}}_{ij}$ which act as selectors for times indices where bus $i$ connects to, charges, and disconnects from a charger during the $j^{\text{th}}$ stop respectively. The on-ramping charging, and off-ramping periods will be used to represent power use from $k^{\text{start}}_{ij}$ to $k^{\text{end}}_{ij} + 1$, $k^{\text{start}}_{ij} + 1$ to $k^{\text{end}}_{ij}$ and $k^{\text{end}}_{ij}$ to $k^{\text{end}}_{ij} + 1$ respectively. 
\par Let $\mathbf{f}$ be a vector of one-based integer indices such that $\mathbf{f}_w = w \ \forall w \in (1,\text{nTime})$. The values in $\mathbf{s}^{\text{up}}_{ij}$ can be defined by the constraint
\begin{equation}\label{eqn:idxStart}\begin{aligned}
	k^{\text{start}}_{ij} &= \mathbf{f}^T\mathbf{s}^{\text{up}}_{ij} \\
	1 &= \mathbf{1}^T\mathbf{s}^{\text{up}}_{ij} \\
	\mathbf{s}^{\text{up}}_{ij} &\in \{0,1\}.
\end{aligned} \end{equation}
Similarly, the values for $\mathbf{s}^{\text{off}}_{ij}$ can be defined as
\begin{equation} \label{eqn:idxEnd}\begin{aligned}
	k^{\text{stop}}_{ij} &= \mathbf{f}^T\mathbf{s}^{\text{off}}_{ij}\\ 
	1 &= \mathbf{1}^T\mathbf{s}^{\text{off}}_{ij} \\
	\mathbf{s}^{\text{off}}_{ij} &\in \{0,1\}.
\end{aligned} \end{equation}

Equations \ref{eqn:idxStart} and \ref{eqn:idxEnd} can be expressed in standard form and zero padded to form additional constraints in terms of $\mathbf{y}$.
\begin{equation} \begin{aligned}
	\begin{bmatrix} 0 & \mathbf{0}^T & -1 & \mathbf{f}^T \\
		        0 & \mathbf{1}^T &  0 & 0            \\
		       -1 & \mathbf{f}^T & 0 & \mathbf{0}^T  \\
		        0 & 0            & 0 & \mathbf{1}^T 
	\end{bmatrix} 
	\begin{bmatrix} k_{ij}^{\text{start}}       \\
		        \mathbf{s}_{ij}^{\text{up}} \\ 
			k_{ij}^{\text{stop}}        \\ 
			\mathbf{s}_{ij}^{\text{off}} 
	\end{bmatrix} &= 
	\begin{bmatrix} 0 \\ 
			1 \\
	                0 \\
			1
	\end{bmatrix} \ \forall i,j \\
	A_{\text{eq3}}\mathbf{y} &= \mathbf{b}_{\text{eq3}}
\end{aligned} \end{equation}
TODO:
\begin{enumerate}
	\item present these constraints in a more systematic way, explaining why they are the way the are.
	\item show how these constraints sandwich the values we desire and force all to be accounted for. 
	\item Include a figure that illustrates this
\end{enumerate}
The final piece is to define $\mathbf{s}^{\text{on}}$. The set of constraints can be described as follows: 
\begin{equation} \label{eqn:idxMiddle}\begin{aligned}
	\mathbf{1}^T\mathbf{s}^{\text{on}}_{ij} &= k^{\text{end}}_{ij} - k^{\text{start}}_{ij} - 1 \\
	s_w\cdot f_w &\le k^{\text{end}}_{ij} + M(1 - s_w) \ \forall s_w \in \mathbf{s}^{\text{on}}_{ij}\\
	s_w\cdot f_w &\ge k^{\text{start}}_{ij} - M(1 - s_w) \ \forall s_w \in \mathbf{s}^{\text{on}}_{ij}\\ 
\end{aligned} \end{equation}
where $M$ is $2\cdot\text{nTime}$.
Equation \ref{eqn:idxMiddle} can also be expressed in standard form as
\begin{equation} \begin{aligned}
	\begin{bmatrix}\mathbf{1}^T & - 1 & 1 \end{bmatrix} \begin{bmatrix}\mathbf{s}_{ij} \\ k_{ij}^{\text{end}} \\ k_{ij}^{\text{start}} \end{bmatrix} = -1
\end{aligned} \end{equation}
and
\begin{equation}\begin{aligned}
	s_w\left (f_w + M \right) - k_{ij}^{\text{end}} &\le M \\
	s_w\left (M - f_w\right) + k_{ij}^{\text{start}} &\le M 
\end{aligned}\end{equation}
which implies that
\begin{equation} \begin{aligned}
	\begin{bmatrix}f_w + M & -1 & 0\\
		       M - f_w & 0 & 1 
	\end{bmatrix} 
	\begin{bmatrix}s_w                 \\
		       k_{ij}^{\text{end}} \\ 
		       k_{ij}^{\text{start}}
	\end{bmatrix} \le
	\begin{bmatrix} M \\
	                M 
	\end{bmatrix} \forall s_w \ \in \mathbf{s}_{ij}^{\text{on}}.
\end{aligned}\end{equation}
The constraints for all $i,j$ are concatenated and zero padded to form a joint matrix which satisfies the constraints and has the form 
\begin{equation}
	A_{\text{ineq6}}\mathbf{y} = \mathbf{b}_{\text{ineq6}}.
\end{equation}
The equality constraint in equation \ref{eqn:idxMiddle} is also standardized such that
\begin{equation} \begin{aligned}
	\mathbf{1}^T\mathbf{s}_{ij}^{\text{on}} - k_{ij}^{\text{end}} + k_{ij}^{\text{start}} &= -1 \ \forall i,j\\
	\begin{bmatrix}\mathbf{1}^T & -1 & 1\end{bmatrix} \begin{bmatrix}\mathbf{s}_{ij}^{\text{on}} \\ k_{ij}^{\text{end}} \\ k_{ij}^{\text{start}} \end{bmatrix} &= -1 \ \forall i,j \\
		A_{\text{eq4}}\mathbf{y} &= \mathbf{b}_{\text{eq4}}
\end{aligned} \end{equation}

\par We next define the average per use for each charger during an interval. The on-ramp intervals can be constrained as
\begin{equation}\begin{aligned}
	p^{\text{on-ramp}}_{ij} &= \frac{p\cdot (\Delta T - r^{\text{start}}_{ij})}{\Delta T}\\
	p^{\text{off-ramp}}_{ij} &= \frac{p\cdot r^{\text{stop}}_{ij}}{\Delta T}\\
\end{aligned}\end{equation}
or in standard form as
\begin{equation}\begin{aligned}
	\begin{bmatrix} 1 & \frac{p}{\Delta T} & 0 & 0                   \\
		        0 & 0                  & 1 & -\frac{p}{\Delta T} \\
	\end{bmatrix}
	\begin{bmatrix} p^{\text{on-ramp}}_{ij}  \\
		        r_{ij}^{\text{start}}    \\
			p^{\text{off-ramp}}_{ij} \\ 
			r^{\text{stop}}_{ij} 
	\end{bmatrix} &=
	\begin{bmatrix} p \\
	                0
	\end{bmatrix} \ \forall i,j \\
	A_{\text{eq5}}\mathbf{y} &= \mathbf{b}_{\text{eq5}}
\end{aligned}\end{equation}
where $p_{ij}^{\text{on-ramp}}$, $p_{ij}^{\text{off-ramp}}$, and $p$ represent the average power for the on-ramping, off-ramping, and charging intervals respectively. The total average power use is calculated as 
\begin{align}\label{eqn:totalPower}
	\mathbf{p}_{\text{total}} = \bar{\mathbf{p}}_{\text{load}} + \sum_{ij} \mathbf{s}^{\text{up}}_{ij}\cdot p^{\text{on-ramp}}_{ij} + \mathbf{s}^{\text{on}}_{ij}\cdot p + \mathbf{s}^{\text{off}}_{ij}\cdot p^{\text{off-ramp}}_{ij}
\end{align}
where $\bar{\mathbf{p}}_{\text{load}}$ is the average power of the uncontrolled loads.
\par Note, however that the results from equation \ref{eqn:totalPower} contain a bilinear term. The first bilnear expression in  equation \ref{eqn:totalPower} can be rewritten as 
\begin{equation} \label{eqn:onPower}\begin{aligned}
	p_w &\ge p^{\text{on-ramp}}_{ij} - M(1 - s_w) \ \forall s_w \in \mathbf{s}^{\text{up}}_{ij}, \ p_w \in \mathbf{p}_{ij}^{\text{up}}\\
	p_w &\le p^{\text{on-ramp}}_{ij} + M(1 - s_w) \ \forall s_w \in \mathbf{s}^{\text{up}}_{ij}, \ p_w \in \mathbf{p}_{ij}^{\text{up}}\\
	p_w &\ge -Ms_w \ \forall s_w \in \mathbf{s}^{\text{up}}_{ij}, \ p_w \in \mathbf{p}_{ij}^{\text{up}}\\
	p_w &\le Ms_w \ \forall s_w \in \mathbf{s}^{\text{up}}_{ij}, \ p_w \in \mathbf{p}_{ij}^{\text{up}}
\end{aligned} \end{equation}
and similarly the second as, 
\begin{equation} \label{eqn:offPower} \begin{aligned}
	p_w &\ge p^{\text{off-ramp}}_{ij} - M(1 - s_w) \ \forall s_w \in \mathbf{s}^{\text{off}}_{ij}, \ p_w \in \mathbf{p}_{ij}^{\text{off}}\\
		p_w &\le p^{\text{off-ramp}}_{ij} + M(1 - s_w) \ \forall s_w \in \mathbf{s}^{\text{off}}_{ij}, \ p_w \in \mathbf{p}_{ij}^{\text{off}}\\
		p_w &\ge -Ms_w \ \forall s_w \in \mathbf{s}^{\text{off}}_{ij}, \ p_w \in \mathbf{p}_{ij}^{\text{off}}\\
		p_w &\le Ms_w \ \forall s_w \in \mathbf{s}^{\text{off}}_{ij}, \ p_w \in \mathbf{p}_{ij}^{\text{off}}.
\end{aligned} \end{equation} 
Equation \ref{eqn:onPower} can be written in standard form, stacked to accomodate the constraints for all $i,j$, and zero padded appropriately as
\begin{equation}\begin{aligned} 
	\begin{bmatrix}
		-1 & 1 & M \\
		1  & -1 & M \\
		-1 & 0 & -M \\
		1 & 0 & -M 
	\end{bmatrix}
	\begin{bmatrix} 
		p_w                     \\
	        p_{ij}^{\text{on-ramp}} \\
		s_w
	\end{bmatrix}  &\le
	\begin{bmatrix}
		M \\
		M \\
		0 \\
		0
	\end{bmatrix} \forall s_w \in \mathbf{s}_{ij}^{\text{up}}, p_w \in \mathbf{p}_{ij}^{\text{off}}.\\ 
	A_{\text{ineq7}} &\le \mathbf{b}_{\text{ineq7}} 
\end{aligned}\end{equation}
	Equation \ref{eqn:offPower} can be expressed in standard form, stacked for all $i,j$, and zero padded in the usual fashion as 
\begin{equation}\begin{aligned} 
	\begin{bmatrix}
		-1 & 1 & M  \\
		1  & -1 & M \\
		-1 & 0 & -M \\
		1 & 0 & -M 
	\end{bmatrix}	
	\begin{bmatrix} p_w                      \\
		        p_{ij}^{\text{off-ramp}} \\
			s_w
	\end{bmatrix} &\le
	\begin{bmatrix} M \\
	                M \\
	                0 \\
	                0
	\end{bmatrix} \forall s_w \in \mathbf{s}_{ij}^{\text{off}}, p_w \in \mathbf{p}_{ij}^{\text{on}} \\
	A_{\text{ineq8}}\mathbf{y} & \le b_{\text{ineq8}}.
\end{aligned}\end{equation} 
An expression for the total power used can then be expressed as
\begin{equation}
	\begin{aligned}
		\mathbf{p}_{\text{total}} = \mathbf{p} + \sum_{ij} \mathbf{p}^{\text{up}}_{ij} + \mathbf{p}^{\text{off}}_{ij} + \mathbf{s}^{\text{on}}_{ij}\cdot p
	\end{aligned}
\end{equation}
and in standard form as
\begin{equation} \begin{aligned}
	\begin{bmatrix}
		1 & -1 & 1_{ij}^{\text{up}} & 1_{ij}^{\text{off}} & 1_{ij}^{\text{on}}\cdot p  
	\end{bmatrix}
	\begin{bmatrix}
		p^{\text{total}}_w            \\
		p_w                           \\
	        \mathbf{p}_{ijw}^{\text{up}}  \\
	        \mathbf{p}_{ijw}^{\text{off}} \\ 
	        \mathbf{s}_{ijw}^{\text{on}}
	\end{bmatrix} &= 0 \\
	A_{\text{eq6}}\mathbf{y} &= \mathbf{b}_{\text{eq6}} 
\end{aligned} \end{equation}

